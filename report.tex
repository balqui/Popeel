%
% MENCIONAR ESTE REPORT EN LA PRESENTACION EN LA WEB ORIGINAL DE BITBUCKET
%
\documentclass[12pt]{article} 
\title{A Software Simulator of the Very Simple Robot \emph{Popeel}%
%\thanks{}
\\ \medskip\Large
(or: Potato Peeling, Resurrected!)
}  
\author{Jos\'e L. Balc\'azar \\
% {\vspace{-0.5cm}} \\ 
{Department of Computer Science} \\
{Universitat Polit\`ecnica de Catalunya}}
\usepackage{fullpage}
\usepackage{hyperref}
\usepackage{underscore}\makeatletter
%\usepackage[ruled,vlined]{algorithm2e} % DOUBTFUL...
%\newtheorem{theorem}{Theorem}
%\newtheorem{lemma}[theorem]{Lemma}
\newtheorem{example}{Example}
% \def\cite#1{[{\sl #1}]} % remove this when the references are ready
\def\jlb#1{\par\noindent\smallskip{\bf\ [#1 --JLB]\par\smallskip}}
% \def\jlb#1{\relax}
% \usepackage{latexsym}
% \usepackage{multicol}
% \usepackage{enumitem}

\begin{document}

\maketitle

\begin{abstract}
We describe an open-source implementation of a
Python class that supports an old programming
exercise related to peeling potatoes.
\end{abstract}


\section{Introduction}% \label{s:intro}

Many years ago, at FIB, UPC \cite{FIB}, we used to 
propose some simple, on-paper programming exercises 
to newcoming students. They involved peeling potatoes, 
one by one, and keeping a count of them, and fetching 
fresh basketfuls of potatoes when necessary. 

Mostly everyone with no previous programming 
experience had some trouble solving the main
statement. Indeed, it happens to be a hardish but 
instructive exercise at that level. However, 
for novel programmers, writing programs on paper 
(or a blackboard) and imagining their runtime effect 
is difficult, and particularly so when the effect 
is a runtime error.

Thus, often, a number of students had trouble simply understanding 
what they were supposed to do \cite{PBblog} (but, what 
potatoes are they talking about?, must I bring an actual 
knife from home tomorrow?\dots) Eventually, we stopped 
proposing this exercise; but, even after such long years,
many students of the time still vaguely remember, 
if not the precise exercise, some approximate
notions of it.

\looseness=-1
These days, however, one can easily provide on the 
web a working implementation, and object orientation
allows for a simple syntax very similar to the 
on-paper exercise of the old times. We have decided
to create one \cite{Popeel}, and expect to use it in the early
lectures of the second edition of a first-year
course on Programming \cite{mireport} in a local 
Degree on Bioinformatics~\cite{BDBI}.

As happened in the old times, in our implementation
there is a toy robot, Popeel, the 
{\bf Po}tato {\bf peel}er, assumed to live in an 
environment where an inexhaustible store of potatoes 
is available nearby. It also has a basket with a few 
potatoes. It can pick one potato from the basket and 
peel it but, of course, only if the basket is nonempty. It can go 
fetch more potatoes from the main store (the \emph{pantry}), 
but will complain 
if asked to do so unnecessarily. The basic operations
offered by the class are to be used in writing small,
correct programs for specific tasks.

Each programming exercise sets up a task for the
robot, that usually includes peeling some potatoes
and that usually ends up putting the robot to sleep,
after having finished its task. However, if asked to
do something impossible, like peeling one potato
from the basket
when the basket is empty, Popeel will complain and
just go sleeping --- further requests for action
when it is sleeping will cause a bitter complain
and then going back to bed.

The reader is welcome to imagine it as a humanoid 
robot but, for the time being, it is just a Python 
class that informs of what is happening through
a monologue.


\section{The main class}

The program is freely available in open-source,
MIT-licensed form \cite{MITlicense} and you are welcome
to download it \cite{Popeel}.
The main class is, of course, {\tt Popeel}: a class
written in Python and compatible with both
Python2 and Python3.

The class is programmed so that it can be used in a course
{\it before} the students have been exposed to the 
notion of a variable (and we are using it so).
It follows the ``singleton'' pattern: after 
importing the name, one just calls the methods as in 
{\tt Popeel.peel_1_potato()} or {\tt Popeel.go_sleep()}
 --- more precisely, they are ``@{\tt classmethod}'''s.

If variables are already conceptually available, then
one object of class {\tt Popeel} 
can be declared as well, and used like with a regular class,
as in, say, {\tt p = Popeel()}. Then, its methods
can be called, as in {\tt p.peel_1_potato()} or
{\tt p.go_sleep()}, following the standard
Python object-oriented syntax; but, being a
singleton class, only one object of the class
will be allowed. 

%\looseness=1
An alternative, ``non-singleton'' class is also 
available: {\tt Popeel_}, with an extra underscore
at the end of its name. This variant can be used in the
fully standard form: as many objects of the class 
as desired can be declared, and their 
methods (now not anymore ``@{\tt classmethod}'''s!) %, but fully standard) 
called in the usual way. However, if several
objects are in use, their messages will be likely
to show up in a confusing manner. The statements
we use {\tt Popeel} for only need one instance, and
we discourage, for the time being, the usage of this 
standard variant of the class.


\subsection{Available operations}

The simple repertory of operations
offered by this class of objects is:

\begin{description}
\item{{\tt peel_1_potato():}} Will peel one potato
from the basket and report about it; however, will
complain and go sleeping if it has no potatoes at all
in the basket.
\item{{\tt refill_basket():}} Will go to the pantry
% main store 
and refill the empty basket with a variable 
quantity of potatoes; will do nothing (except complaining)
if it still has potatoes in the basket. 
\item{{\tt discard_basket():}} Will return potatoes
left in the basket back to the pantry. Popeel likes
to do this when its task is complete. 
\item{{\tt go_sleep():}} Popeel will complain if you
wake it up after going to bed --- and then return to
bed anyhow.
\item{{\tt set_task(int):}} The goal is now to peel
the number of potatoes indicated as argument.
There is a default, implicit task for programs 
that do not call this method.
\item{{\tt is_basket_empty():}} Boolean function, 
result as by its name.
\item{{\tt enough_potatoes():}} Boolean function, indicates
whether the task set (either explicitly through {\tt set_task}
or implicitly) has been accomplished. 
%Recommended usage before sending Popeel to bed. 
\end{description}

Somewhere in the download, we provide file 
{\tt popeel_user.py} 
with example usages of these operations.

% Some of these include {\tt print} statements, 
% which may become unnecessary if the
% operations are run within the read-eval-print loop of
% an interpreter.

\section{Example tasks}

There is a main exercise task, that we leave for the 
next section, as
% where we leave the
% goal quantity of potatoes to the initialization,
% which will set up {\tt Popeel} in a somewhat random state;
% we return to this in the next section. However, 
it
is worthwhile to start with simpler tasks. All
the rest of this section assumes that you have
a Python interpreter available and that the
source code for the class, {\tt popeel.py}, is within
the appropriate system paths (most likely, it will
suffice to download it to some folder in your 
equipment and run Python on program files in
the same folder).

\subsection{Peeling just one potato (and then taking a rest)}

It would seem that peeling one potato is to be
the simplest task to program on top of Popeel,
but actually a subsequent one is easier, as we
shall see.

% We always start with a declaration of an object
% of the class {\tt Popeel}, after importing the class
% to have it available to our program:

% \qquad{\tt from popeel import Popeel}

% \qquad{\tt p = Popeel()}

% (The identifier {\tt p} is arbitrary, any other
% name will do; but use it consistently later on.)

% One might guess that the subsequent instructions
% would be:

% \qquad{\tt p.peel_1_potato()}

% \qquad{\tt p.go_sleep()}

Indeed, one might guess that this program would
accomplish the task:

\qquad{\tt from popeel import Popeel}

\qquad{\tt Popeel.peel_1_potato()}

\qquad{\tt Popeel.go_sleep()}

You might see on your own why this solution is not correct; 
otherwise, try it, insistingly, maybe even many times, 
before reading further. 

For your convenience, we provide 
this program as file 
{\tt popeel_simple_user.py} somewhere in the download.

% By the way, the simpler, quite intuitive, but equally wrong 
% alternative using the singleton class is:
% \qquad{\tt from popeel import Popeel_}
% \qquad{\tt Popeel_.peel_1_potato()}
% \qquad{\tt Popeel_.go_sleep()}
%
Once you understand why this solution is not
correct, try and come up with a correct one
on your own. 

The task of peeling two potatoes is postponed for the moment.

\subsection{Peeling all the potatoes in the basket}

In this exercise, we task Popeel with peeling a
variable number of potatoes: whatever ones it 
happened to have in the basket upon being summoned,
which will be randomly chosen below a certain limit.
Thus, each run of this program will end up peeling a 
different quantity of potatoes. Of course, once the
basket becomes empty, Popeel is to be sent to sleep.

% For your convenience, we provide this program as file 
% {\tt popeel_first_user.py} but we heartily recommend to
% try 
Try your hand at writing the program without looking
inside any file in the download. We believe it will 
be easier than it seems.

\subsection{Peeling two potatoes}

The section title says it all. 
Write your own solution and test it
repeatedly and insistently before labeling it
as good! 

Don't forget sending Popeel to bed.
Consider also the option of saving food by
returning left-over potatoes, which remain
unused in the basket, back to the pantry.

\subsection{Peeling two basketfuls of potatoes}

Again each run of this program will end up peeling a 
different quantity of potatoes, but this time, once
all the potatoes in the initial basket are peeled,
we refill the basket and peel all the potatoes again.
% And again, for your convenience, we provide a solution 
% as file {\tt popeel_second_user.py} but we insist that you
% should write and thoroughly test your program before 
% comparing it to our own.

% \subsection{Random Popeel programs}

% The download includes a number of programs created
% randomly, as well as a program, {\tt randprog.py}, that
% you can use yourself to create random programs on 
% top of Popeel\dots\ and to guess what they do \emph{before}
% running them (but there is no need of your looking 
% inside {\tt randprog.py} --- its essence is a so-called
% probabilistic generative grammar).

\section{The main exercise}

The main exercise is to peel a fixed amount of potatoes.
There is a default value (currently 35 potatoes) 
which you of course override by
calling {\tt Popeel.set_task(12)} if you want 
to peel 12 potatoes, or whatever number you fancy. 
Alternatively, use {\tt p.set_task(12)} if
you declare an object {\tt p} of class {\tt Popeel}.

Thus, say the program starts with:

\qquad{\tt from popeel import Popeel}

% \qquad{\tt p = Popeel()}

\qquad{\tt Popeel.set_task(12)}

How does it continue? We do not provide a solution:
it is your task to construct it, and we hope that
some readers will have learned something by doing
this exercise. We do provide, however, a couple of
examples of Popeel's monologue for two correct cases:
in one, it happens to start with no potatoes in the
basket (file {\tt popeel_correct_output_0.txt});
in the other, bearing in fact little difference,
it happens to start with some potatoes 
in the
basket (file {\tt popeel_correct_output_6.txt}).
Of course, the exact quantities, and the places where
Popeel reports the number of potatoes peeled so far,
or needs to fetch more potatoes,
may vary, depending on the (somewhat random) 
quantities of potatoes
obtained at each time of refilling the basket.

\begin{thebibliography}{10}

\bibitem{FIB}
\url{www.fib.upc.edu}

\bibitem{PBblog}
\url{https://patriciamiriam.wordpress.com/2017/02/14/carta-a-mi-tio-en-respuesta-a-la-suya-del-23-de-noviembre-de-1982/}

\bibitem{Popeel}
\url{https://bitbucket.org/balqui/popeel}

\bibitem{mireport}
\url{http://www.cs.upc.edu/~balqui/proal16.pdf}

\bibitem{BDBI}
\url{https://www.esci.upf.edu/en/bachelors-degree-in-bioinformatics/bachelors-degree-bioinformatics}

\bibitem{MITlicense}
\url{https://en.wikipedia.org/wiki/MIT_License}

\end{thebibliography}

\end{document}











