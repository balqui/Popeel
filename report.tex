\documentclass[12pt]{article} 
\title{A Software Simulator of the Very Simple Robot POPEEL%
%\thanks{}
}  
\author{Jos\'e L. Balc\'azar \\
% {\vspace{-0.5cm}} \\ 
{Department of Computer Science} \\
{Universitat Polit\`ecnica de Catalunya}}
\usepackage{fullpage}
\usepackage{hyperref}
%\usepackage[ruled,vlined]{algorithm2e} % DOUBTFUL...
%\newtheorem{theorem}{Theorem}
%\newtheorem{lemma}[theorem]{Lemma}
\newtheorem{example}{Example}
% \def\cite#1{[{\sl #1}]} % remove this when the references are ready
\def\jlb#1{\par\noindent\smallskip{\bf\ [#1 --JLB]\par\smallskip}}
% \def\jlb#1{\relax}
% \usepackage{latexsym}
% \usepackage{multicol}
% \usepackage{enumitem}

\begin{document}

\maketitle

\begin{abstract}
We describe an open-source implementation of a
Python class that supports an old programming
exercise.
\end{abstract}


\section{Introduction}% \label{s:intro}

Many years ago, at FIB, UPC, we used to propose some 
simple on-paper programming exercises to newcoming 
students. They involved peeling potatoes, one by one, 
and keeping a count of them, and fetching fresh 
basketfuls of potatoes when necessary. 

Mostly everyone had some trouble solving the main
statement, because it happens to be a hardish but 
instructive exercise for people with no programming 
experience. However, many of them had additional 
trouble simply understanding what they were supposed 
to do (but, where are the potatoes?, must I bring 
an actual knife from home?\dots) Eventually, we
stopped proposing this exercise.



